\documentclass[xcolor=table]{beamer}
% generated by Madoko, version 1.0.3
%mdk-data-line={1}


\usepackage[heading-base={2},section-num={False},bib-label={True}]{madoko2}
%mdk-data-line={1;out\presentation.mdk:79}

    \ifbeamer\relax\else
      \providecommand{\usetheme}[2][]{}
      \providecommand{\usecolortheme}[2][]{}
      \providecommand{\usefonttheme}[2][]{}
      \providecommand{\pause}[1][]{}
      \providecommand{\AtBeginSection}[2][]{}
      \providecommand{\AtBeginSubsection}[2][]{}
      \providecommand{\AtBeginSubsubsection}[2][]{}
      \providecommand{\AtBeginPart}[2][]{}
      \providecommand{\AtBeginLecture}[2][]{}
      \providecommand{\theoremstyle}[2][]{}
      \makeatletter
      \def\newtheorem{\@ifstar\newtheoremx\newtheoremx}
      \makeatother
      \providecommand{\newtheoremx}[3][]{}{}
    \fi
%mdk-data-line={1;out\presentation.mdk:96}

    \ifbeamer\usetheme[]{singapore}\fi
\begin{document}



%mdk-data-line={10}
%mdk-data-line={10}
%mdk-data-line={11}
\mdxtitleblockstart{}
%mdk-data-line={11}
\mdxtitle{{\mdfontsize{3em}\mdline{11}数独浅谈}}%mdk

%mdk-data-line={14}
\mdxsubtitle{{\mdfontsize{1.8em}\mdline{14}古老的数字逻辑游戏}}%mdk
\mdxauthorstart{\Large{}
%mdk-data-line={19}
\mdxauthorname{{\mdfontsize{1.4em}\mdline{19}符晓浛}}%mdk

%mdk-data-line={22}
\mdxauthoraddress{\mdline{22}上海交通大学}%mdk

%mdk-data-line={25}
\mdxauthoremail{\mdline{25}reapor.yurnero@sjtu.edu.cn}%mdk
}\mdxauthorend\mdtitleauthorrunning{}{}\mdxtitleblockend%mdk

%mdk-data-line={12}
\begin{mdframe}%mdk

\frametitle{标准数独}\label{heading-section}%mdk%mdk

%mdk-data-line={14}
\begin{mdcenter}%mdk

%mdk-data-line={16}
\noindent\mdline{16}\includegraphics[keepaspectratio=true]{https:/upload.wikimedia.org/wikipedia/commons/thumb/f/ff/Sudoku-by-L2G-20050714.svg/364px-Sudoku-by-L2G-20050714.svg}{}\mdline{16}%mdk

%mdk-data-line={17}
\mdhr{}%mdk

%mdk-data-line={18}
\noindent\mdline{18}上图即为一个典型的标准数独%mdk
%mdk
\end{mdcenter}%mdk
%mdk
\end{mdframe}\label{section}%mdk%mdk

%mdk-data-line={22}
\begin{mdframe}%mdk

\frametitle{数独是什么}\label{heading-section}%mdk%mdk

%mdk-data-line={24}
\noindent\mdline{24}数独 (日语:数独/すうどく Sūdoku)是一种逻辑性的数字填充游戏,玩家须以数字填进每一格,而每行、每列和每个宫(即3x3的大格)有齐1至9所有数字。游戏设计者会提供一部分的数字,使谜题只有一个答案。
一个已解答的数独其实是一种多了宫的限制的拉丁方阵,因为同一个数字不可能在同一行、列或宫中出现多于一次。%mdk
%mdk
\end{mdframe}\label{section}%mdk%mdk

%mdk-data-line={28}
\begin{mdframe}%mdk

\frametitle{数独的起源}\label{heading-section}%mdk%mdk

%mdk-data-line={30}
\noindent\mdline{30}数独起源于18世纪初瑞士数学家欧拉等人研究的拉丁方阵(Latin Square)。
19世纪80年代,一位美国的退休建筑师格昂斯(Howard Garns)根据这种拉丁方阵发明了一种填数趣味游戏,这就是数独的雏形。20世纪70年代,人们在美国纽约的一本益智杂志《Math Puzzles and Logic Problems》上发现了这个游戏,当时被称为填数字(Number Place),这也是目前公认的数独最早的见报版本。1984年一位日本学者将其介绍到了日本,发表在Nikoli公司的一本游戏杂志上,当时起名为“Suuji wa dokushin ni kagiru”,就改名为“sudoku”,其中“su”是数字的意思,“doku”是单一的意思。后来一位前任香港高等法院的新西兰籍法官高乐德(Wayne Gould)在1997年3月到日本东京旅游时,无意中发现了。
他首先在英国的《泰晤士报》上发表,不久其他报纸也发表,很快便风靡全英国。随着计算机信息技术的发展,数独开始在全世界流行。%mdk
%mdk
\end{mdframe}\label{section}%mdk%mdk

%mdk-data-line={34}
\begin{mdframe}%mdk

\frametitle{规则}\label{heading-section}%mdk%mdk

%mdk-data-line={36}
\noindent\mdline{36}数独的规则非常简单:%mdk

%mdk-data-line={38}
\begin{itemize}[noitemsep,topsep=\mdcompacttopsep]%mdk

%mdk-data-line={38}
\item\mdline{38}每一行由1\mdline{38}\textasciitilde{}\mdline{38}9单独出现一次%mdk

%mdk-data-line={39}
\item\mdline{39}每一列由1\mdline{39}\textasciitilde{}\mdline{39}9单独出现一次%mdk

%mdk-data-line={40}
\item\mdline{40}每一个小的九宫格1\mdline{40}\textasciitilde{}\mdline{40}9由单独出现一次%mdk
%mdk
\end{itemize}%mdk
%mdk
\end{mdframe}\label{section}%mdk%mdk

%mdk-data-line={44}
\begin{mdframe}%mdk

\frametitle{解谜技巧}\label{heading-section}%mdk%mdk

%mdk-data-line={45}
\noindent\mdline{45}由上述的三条基本规则,我们可以衍生出摸索出一些简单实用的技巧。学习这些技巧,能有效的缩短解题的时间。\mdline{45}%mdk

%mdk-data-line={47}
\mdline{47}不过大家要始终牢记,所有的方法都来源于规则,因此,没有必要拘泥于一些所谓的方法,这些方法只是一些辅助,深入理解规则才是王道。%mdk

%mdk-data-line={49}
\begin{mdcenter}%mdk

%mdk-data-line={50}
\noindent\mdline{50}\includegraphics[keepaspectratio=true]{https:/upload.wikimedia.org/wikipedia/commons/thumb/f/ff/Sudoku-by-L2G-20050714.svg/364px-Sudoku-by-L2G-20050714.svg}{}%mdk
%mdk
\end{mdcenter}%mdk
%mdk
\end{mdframe}\label{section}%mdk%mdk

%mdk-data-line={54}
\begin{mdframe}%mdk

\frametitle{单向扫看法}\label{heading-section}%mdk%mdk

%mdk-data-line={55}
\begin{mdcenter}%mdk

%mdk-data-line={56}
\noindent\mdline{56}如下图所示,我们可以利用这种方法迅速地找出一些数字的正确位置。%mdk
%mdk
\end{mdcenter}%mdk
\begin{mdtabular}{2}{\dimeval{(\linewidth-\dimwidth{0.50})/1}}{0pt}%mdk
\begin{tabular}{ll}

%mdk-data-line={59}
\begin{mdcolumn}%mdk
\begin{mdblock}{width=\dimwidth{0.50}}%mdk
%mdk-data-line={60}
\noindent\mdline{60}\includegraphics[keepaspectratio=true,width=\dimpx{480}]{http:/www.conceptispuzzles.com/zh/picture/27/1186}{}\mdline{60}%mdk%mdk
\end{mdblock}%mdk
\end{mdcolumn}%mdk
&
%mdk-data-line={63}
\begin{mdcolumn}%mdk
\begin{mdblock}{width=\dimavailable}%mdk
%mdk-data-line={64}
\noindent\mdline{64}\includegraphics[keepaspectratio=true,width=\dimpx{480}]{http:/www.conceptispuzzles.com/zh/picture/27/1187}{}\mdline{64}%mdk%mdk
\end{mdblock}%mdk
\end{mdcolumn}%mdk
\\
\end{tabular}\end{mdtabular}
%mdk
\end{mdframe}\label{section}%mdk%mdk

%mdk-data-line={70}
\begin{mdframe}%mdk

\frametitle{双向向扫看法}\label{heading-section}%mdk%mdk

%mdk-data-line={71}
\begin{mdcenter}%mdk

%mdk-data-line={72}
\noindent\mdline{72}如下图所示,这种方法是单向法的变形。这两者是最常用的方法。%mdk
%mdk
\end{mdcenter}%mdk
\begin{mdtabular}{2}{\dimeval{(\linewidth-\dimwidth{0.50})/1}}{0pt}%mdk
\begin{tabular}{ll}

%mdk-data-line={75}
\begin{mdcolumn}%mdk
\begin{mdblock}{width=\dimwidth{0.50}}%mdk
%mdk-data-line={76}
\noindent\mdline{76}\includegraphics[keepaspectratio=true,width=\dimpx{480}]{http:/www.conceptispuzzles.com/zh/picture/27/1186}{}\mdline{76}%mdk%mdk
\end{mdblock}%mdk
\end{mdcolumn}%mdk
&
%mdk-data-line={79}
\begin{mdcolumn}%mdk
\begin{mdblock}{width=\dimavailable}%mdk
%mdk-data-line={80}
\noindent\mdline{80}\includegraphics[keepaspectratio=true,width=\dimpx{480}]{http:/www.conceptispuzzles.com/zh/picture/27/1187}{}\mdline{80}%mdk%mdk
\end{mdblock}%mdk
\end{mdcolumn}%mdk
\\
\end{tabular}\end{mdtabular}
%mdk
\end{mdframe}\label{section}%mdk%mdk

%mdk-data-line={85}
\begin{mdframe}%mdk

\frametitle{寻找候选法}\label{heading-section}%mdk%mdk

%mdk-data-line={86}
\begin{mdcenter}%mdk

%mdk-data-line={87}
\noindent\mdline{87}这是一种稍进阶的方法,常用来打破僵局。%mdk
%mdk
\end{mdcenter}%mdk
\begin{mdtabular}{2}{\dimeval{(\linewidth-\dimwidth{0.50})/1}}{0pt}%mdk
\begin{tabular}{ll}

%mdk-data-line={90}
\begin{mdcolumn}%mdk
\begin{mdblock}{width=\dimwidth{0.50}}%mdk
%mdk-data-line={91}
\noindent\mdline{91}\includegraphics[keepaspectratio=true,width=\dimpx{480}]{http:/www.conceptispuzzles.com/zh/picture/27/1190}{}\mdline{91}%mdk%mdk
\end{mdblock}%mdk
\end{mdcolumn}%mdk
&
%mdk-data-line={94}
\begin{mdcolumn}%mdk
\begin{mdblock}{width=\dimavailable}%mdk
%mdk-data-line={95}
\noindent\mdline{95}\includegraphics[keepaspectratio=true,width=\dimpx{480}]{http:/www.conceptispuzzles.com/zh/picture/27/1191}{}\mdline{95}%mdk%mdk
\end{mdblock}%mdk
\end{mdcolumn}%mdk
\\
\end{tabular}\end{mdtabular}
%mdk
\end{mdframe}\label{section}%mdk%mdk

%mdk-data-line={100}
\begin{mdframe}%mdk

\frametitle{数字排除法}\label{heading-section}%mdk%mdk

%mdk-data-line={101}
\begin{mdcenter}%mdk

%mdk-data-line={102}
\noindent\mdline{102}这是一种更进阶的方法,当你无路可走时,不妨采用此方法做一尝试。%mdk
%mdk
\end{mdcenter}%mdk
\begin{mdtabular}{2}{\dimeval{(\linewidth-\dimwidth{0.50})/1}}{0pt}%mdk
\begin{tabular}{ll}

%mdk-data-line={105}
\begin{mdcolumn}%mdk
\begin{mdblock}{width=\dimwidth{0.50}}%mdk
%mdk-data-line={106}
\noindent\mdline{106}\includegraphics[keepaspectratio=true,width=\dimpx{480}]{http:/www.conceptispuzzles.com/zh/picture/27/1192}{}\mdline{106}%mdk%mdk
\end{mdblock}%mdk
\end{mdcolumn}%mdk
&
%mdk-data-line={109}
\begin{mdcolumn}%mdk
\begin{mdblock}{width=\dimavailable}%mdk
%mdk-data-line={110}
\noindent\mdline{110}\includegraphics[keepaspectratio=true,width=\dimpx{480}]{http:/www.conceptispuzzles.com/zh/picture/27/1193}{}\mdline{110}%mdk%mdk
\end{mdblock}%mdk
\end{mdcolumn}%mdk
\\
\end{tabular}\end{mdtabular}

%mdk
\end{mdframe}\label{section}%mdk%mdk

%mdk-data-line={152}
\begin{mdframe}%mdk

\frametitle{实战时间}\label{heading-section}%mdk%mdk

%mdk-data-line={154}
\begin{mdcenter}%mdk

%mdk-data-line={155}
\noindent\mdline{155}请大家尝试尽快完成此题,最快完成正确作答的同学,将获取精美礼品一份%mdk

%mdk-data-line={156}
\mdhr{}%mdk

%mdk-data-line={157}
\noindent\mdline{157}\includegraphics[keepaspectratio=true]{https:/upload.wikimedia.org/wikipedia/commons/thumb/f/ff/Sudoku-by-L2G-20050714.svg/364px-Sudoku-by-L2G-20050714.svg}{}\mdline{157}%mdk

%mdk-data-line={158}
\mdhr{}%mdk
%mdk
\end{mdcenter}%mdk
%mdk
\end{mdframe}\label{section}%mdk%mdk

%mdk-data-line={162}
\begin{mdframe}%mdk

\frametitle{特种数独}\label{heading-section}%mdk%mdk

%mdk-data-line={163}
\noindent\mdline{163}在标准数独之外,还有一些后来者发明的更有趣更复杂的数独,例如杀手数独、对角线数独等,我将在此为大家略作分享。%mdk

%mdk-data-line={165}
\mdline{165}但无论千变万化,数独的基本规则始终不变,只是在此基础上有所补充。%mdk
%mdk
\end{mdframe}\label{section}%mdk%mdk

%mdk-data-line={169}
\begin{mdframe}%mdk

\frametitle{变形数独}\label{heading-section}%mdk%mdk

%mdk-data-line={170}
\begin{mdcenter}%mdk

%mdk-data-line={171}
\noindent\mdline{171}原先9x9中的小九宫格变成了不规则的形状,规则也随之改变。%mdk
%mdk
\end{mdcenter}%mdk
\begin{mdtabular}{2}{\dimeval{(\linewidth-\dimwidth{0.50})/1}}{0pt}%mdk
\begin{tabular}{ll}

%mdk-data-line={175}
\begin{mdcolumn}%mdk
\begin{mdblock}{width=\dimwidth{0.50}}%mdk
%mdk-data-line={176}
\noindent\mdline{176}\includegraphics[keepaspectratio=true,width=\dimpx{480}]{https:/upload.wikimedia.org/wikipedia/commons/thumb/5/59/A_nonomino_sudoku.svg/225px-A_nonomino_sudoku.svg}{}\mdline{176}%mdk%mdk
\end{mdblock}%mdk
\end{mdcolumn}%mdk
&
%mdk-data-line={179}
\begin{mdcolumn}%mdk
\begin{mdblock}{width=\dimavailable}%mdk
%mdk-data-line={180}
\noindent\mdline{180}\includegraphics[keepaspectratio=true,width=\dimpx{480}]{https:/upload.wikimedia.org/wikipedia/commons/thumb/3/38/A_nonomino_sudoku_solution.svg/460px-A_nonomino_sudoku_solution.svg}{}\mdline{180}%mdk%mdk
\end{mdblock}%mdk
\end{mdcolumn}%mdk
\\
\end{tabular}\end{mdtabular}
%mdk
\end{mdframe}\label{section}%mdk%mdk

%mdk-data-line={185}
\begin{mdframe}%mdk

\frametitle{对角线数独}\label{heading-section}%mdk%mdk

%mdk-data-line={186}
\begin{mdcenter}%mdk

%mdk-data-line={187}
\noindent\mdline{187}对角线数独在标准数独的基础上,增加了主对角线也要有1\mdline{187}\textasciitilde{}\mdline{187}9单次出现的要求,难度上升。%mdk

%mdk-data-line={189}
\mdline{189}\includegraphics[keepaspectratio=true,width=\dimpx{500}]{https:/imgsa.baidu.com/baike/w/%253D268/sign=414ddf96cd95d143da76e3254bf28296/f9198618367adab4490728d68ed4b31c8601e44b
}{}%mdk
%mdk
\end{mdcenter}%mdk
%mdk
\end{mdframe}\label{section}%mdk%mdk

%mdk-data-line={192}
\begin{mdframe}%mdk

\frametitle{杀手数独}\label{heading-section}%mdk%mdk

%mdk-data-line={194}
\noindent\mdline{194}最经典的特种数独之一,在标准数独的规则基础之上,还规定了指定形状区域内的数字之和,难度较大。%mdk
\begin{mdtabular}{2}{\dimeval{(\linewidth-\dimwidth{0.50})/1}}{0pt}%mdk
\begin{tabular}{ll}

%mdk-data-line={197}
\begin{mdcolumn}%mdk
\begin{mdblock}{width=\dimwidth{0.50}}%mdk
%mdk-data-line={198}
\noindent\mdline{198}\includegraphics[keepaspectratio=true,width=\dimpx{480}]{https:/upload.wikimedia.org/wikipedia/commons/thumb/5/5e/Killersudoku_color.svg/408px-Killersudoku_color.svg}{}\mdline{198}%mdk%mdk
\end{mdblock}%mdk
\end{mdcolumn}%mdk
&
%mdk-data-line={201}
\begin{mdcolumn}%mdk
\begin{mdblock}{width=\dimavailable}%mdk
%mdk-data-line={202}
\noindent\mdline{202}\includegraphics[keepaspectratio=true,width=\dimpx{480}]{https:/upload.wikimedia.org/wikipedia/commons/thumb/8/81/Killersudoku_color_solution.svg/408px-Killersudoku_color_solution.svg}{}\mdline{202}%mdk%mdk
\end{mdblock}%mdk
\end{mdcolumn}%mdk
\\
\end{tabular}\end{mdtabular}
%mdk
\end{mdframe}\label{section}%mdk%mdk

%mdk-data-line={207}
\begin{mdframe}%mdk

\frametitle{超数独}\label{heading-section}%mdk%mdk

%mdk-data-line={208}
\begin{mdcenter}%mdk

%mdk-data-line={209}
\noindent\mdline{209}可能是最受欢迎的特种数独,难度很大。在标准数独的基础上,要求额外的九宫格满足条件。%mdk
%mdk
\end{mdcenter}%mdk
\begin{mdtabular}{2}{\dimeval{(\linewidth-\dimwidth{0.50})/1}}{0pt}%mdk
\begin{tabular}{ll}

%mdk-data-line={213}
\begin{mdcolumn}%mdk
\begin{mdblock}{width=\dimwidth{0.50}}%mdk
%mdk-data-line={214}
\noindent\mdline{214}\includegraphics[keepaspectratio=true,width=\dimpx{480}]{https:/upload.wikimedia.org/wikipedia/commons/thumb/1/12/Oceans_Hypersudoku18_Puzzle.svg/225px-Oceans_Hypersudoku18_Puzzle.svg}{}\mdline{214}%mdk%mdk
\end{mdblock}%mdk
\end{mdcolumn}%mdk
&
%mdk-data-line={217}
\begin{mdcolumn}%mdk
\begin{mdblock}{width=\dimavailable}%mdk
%mdk-data-line={218}
\noindent\mdline{218}\includegraphics[keepaspectratio=true,width=\dimpx{480}]{https:/upload.wikimedia.org/wikipedia/commons/thumb/1/17/Oceans_Hypersudoku18_Solution.svg/1000px-Oceans_Hypersudoku18_Solution.svg}{}\mdline{218}%mdk%mdk
\end{mdblock}%mdk
\end{mdcolumn}%mdk
\\
\end{tabular}\end{mdtabular}
%mdk
\end{mdframe}\label{section}%mdk%mdk

%mdk-data-line={225}
\begin{mdframe}%mdk

\frametitle{感谢您的聆听!}\label{heading-section}%mdk%mdk

%mdk
\end{mdframe}\label{section}%mdk%mdk%mdk%mdk%mdk


\end{document}
